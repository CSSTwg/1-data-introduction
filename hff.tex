\documentclass[12pt]{article}
\usepackage{url}
\usepackage{natbib}
\usepackage{threeparttable}
\usepackage{geometry}
\usepackage{graphicx}
\usepackage{tabularx}
\usepackage{caption}
\usepackage[font=footnotesize,labelfont=bf]{caption}

\title{ \bf \huge Data description}
\geometry{left=2.5cm,right=2.5cm,top=2.5cm,bottom=2.5cm}
%\author{\large Ma Chao \\ \\ Kavli Institute for Astronomy and Astrophysics, Peking University}
\date{}
\begin{document}
\maketitle

Galaxy clusters are the largest and most massive gravitationally bound systems in the Universe.
Environmental processes in dense cluster environments  play a crucial role in shaping the stellar population configuration within galaxies, i.e., galaxy morphology, which is one of the most basic observational properties of the galaxies.
Therefore, galaxy clusters serve as critical cosmic laboratories to probe a myriad of science topics relevant to galaxy research.
It is promising that intermedia-redshift clusters will provide us novel insight of environment- dependent galaxy evolution that can not be seen in current epoch.
However, obtaining the detailed structural information of those distant cluster galaxies is impossible with current ground-based instruments.
The high-resolution Hubble Space Telescope (HST) imaging with deep observation has the capability to allow us to access their structural information in detail.

We utilize the deep, high-resolution imaging  from Hubble Frontier Fields (HFF) program to perform the analysis.
The primary science goal of HFF is to push HST observational limit to more distant universe in order to detect very faint, high-redshift galaxies, by combining the power of high-resolution HST with the cosmic gravitational telescope effect of massive high-magnification clusters. In other words, those foreground clusters act as a `magnifying' glass to bring extremely faint galaxies into view such that they can be detected with HST observations.
This program presents a great investment in imaging six strong lensing, extremely massive clusters at intermedia redshifts spanning 0.3$<$z$<$0.55, namely Abell 2744, Abell S1063, Abell 370, MACS J0416.1-2403, MACS J0717.5+3745 and MACS J1149.5+2223. 
Using Director's Discretionary Time, HFF has obtained deepest observations to date of galaxy clusters, along with corresponding parallel deep blank fields offset ($\sim$ 6') from each cluster.
With a total of 140 orbits devoted to each cluster, the high-quality,  multi-band imaging of HFF have revealed unprecedented details of galactic structures for cluster galaxies, which offer us a unique opportunity to study the cluster environmental effects on galaxies properties at such epoch. 
The top row of Figure 1 show the color composite images of six massive lensing clusters, as well as the corresponding parallel fields in the bottom row.
For each cluster and parallel field, the deep, multi-wavelength HFF imaging are obtained with three optical ACS/WFC bands in F435W, F606W and F814W, and four near-infrared WFC3/IR bands including F105W, F125W, F140W and F160W.
They span a wavelength range from $\sim$ 3500 {\AA} to 17400 {\AA}, which roughly corresponds to the rest-frame bands $u$ to $J$ at the redshift of M0416.

\begin{figure}[h]
\begin{center}
\includegraphics[width=1.0\columnwidth]{six_cluster_parallel.png}
\caption{\footnotesize A gallery of Hubble Frontier Fields. Top  panels show six massive galaxy clusters at intermedia redshifts, observed by HST in optical and near-infrared light. From left to right:  Abell 2744, MACS J0416.1-2403, MACS J0717.5+3745, MACS J1149.5+2223, Abell S1063, Abell 370. Bottom panels are associated parallel fields adjacent to the cluster fields. These parallel fields were imaged with the same exposure times as the clusters themselves.    }
\end{center}
\end{figure}

In our analysis, we target on one of the HFF clusters, MACSJ0416.1-2403 (hereafter M0416).
M0416 is a massive X-ray luminous cluster composed of two sub-clusters (joined by Intra-cluster light) undergoing a merging event, at a redshift of 0.396, with a weak lensing mass of $M_{200}=(1.04\pm0.22) \times 10^{15}  {\rm M}_{\odot}$.
We downloaded and analyzed the multi-band science and associated weight images on the HFF program website \footnote{https://archive.stsci.edu/pub/hlsp/frontier/} released by the Space Telescope Science Institute (STScI), and were latest v1.0 mosaics calibrated and reduced by HFF team.
The ACS images has been calibrated with a new `self-calibration' approach to reduce low-level dark current artefacts across the detector, and for the WFC3 images  we chose those corrected for time-variable sky background.
These new released mosaics have been stacked and drizzled at 30 and 60 mas pixel scales (0.06$"$/pix), respectively.
We used the drizzled science images and their associated weight maps with a pixel scale of 0.06$"$. 
For a detailed description of observation strategy and data reduction process, we refer the reader to the STScI data release documentation \footnote{https://archive.stsci.edu/pub/hlsp/frontier/}.

Figure 2 displays the multi-band science imaging of M0416 we used in this work. 
The images in all filters were already aligned onto a uniform astrometric grid and have the same size and pixel scale.
Top-left panel shows the composite RGB image made with filters F160W (red), F125W (green) and F814W (blue).
Table 1 summarizes the basic properties of these multi-wavelength data.  The table lists, for each filter, the  pivot wavelength, the FWHM of point spread function (PSF), exposure time, and the 5$\sigma$  limiting magnitude for point source, which has been calculated from the average pixel variance within an aperture of one FWHM radius (without aperture correction).
By visually inspecting our dataset (Figure 2), the galaxies tend to be more extended in redder bands.
This indicates that the fields become more crowded at longer wavelength. 
In our work, we first develop a series of methods to analysis reddest (most crowded) F160W band. 
Next, we extend these single-band methods to multiple bands of the same cluster.

\begin{figure}[h]
\begin{center}
\includegraphics[width=1.0\columnwidth]{seven_filters.jpg}
\caption{\footnotesize Original images of HFF M0416 cluster in the seven filters. The field sizes have $2838 \times 2909$ pixels with a pixel size of $0.06"$. The top left panel shows the False-color RGB image, constructed using the filters F160W, F125W, and F814W in the red, green, and blue image channel, respectively.}
\end{center}
\end{figure}

\begin{table}
\begin{threeparttable}
\begin{center}
%\begin{minipage}{20mm}
\caption{ Summary of observational filter set}
\begin{tabular}{lcccc}
\hline
\hline \\[-1.4ex]
Instrument/Filter&Pivot Wavelength&PSF FWHM&Exposure Time&Limiting Magnitude\\[0.2ex]
&($\AA$)& ($"$)&(ks)&(5$\sigma$ depth)\\
\hline\\[-1.4ex]
ACS/F435W&4327&0.11&45&28.86\\ 
ACS/F606W&5922&0.11&25&28.97\\
ACS/F814W&8059&0.13&105&29.31\\
WFC3/F105W&10551&0.18&60&29.22\\
WFC3/F125W&12486&0.19&30&28.90\\
WFC3/F140W&13923&0.19&25&28.95\\
WFC3/F160W&15370&0.20&60&29.01\\
\hline
\end{tabular}
\begin{tablenotes}
\item {\bf \footnotesize Notes}. \footnotesize The 5$\sigma$ limiting magnitudes have been measured as $m_{\rm lim}=-2.5 \times {\rm log}_{10}( 5\times \sqrt{N \times \langle \sigma^2 \rangle})$ + zpt,  where N is the total number of pixel set defining the circular aperture (one FWHM radius), $\langle \sigma^2 \rangle$ is the average of the pixel rms squared within the aperture,  and zpt is the zero point of  corresponding filter. 
In each filter, the fiducial value is derived by averaging 1000 apertures scatted in random positions across the field.
\end{tablenotes}
%\end{minipage}
\end{center}
\end{threeparttable}
\end{table}




\end{document}
